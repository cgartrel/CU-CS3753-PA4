% CSCI3753 - Operating Systems
% Spring 2013
% Programming Assignment 4
% By Andy Sayler (4/18/12) <www.andysayler.com>

\documentclass[12pt]{article}

\usepackage[text={6.5in, 9in}, centering]{geometry}
\usepackage{graphicx}
\usepackage{url}
\usepackage{listings}
\usepackage{hyperref}

\lstset{
  language={c},
  basicstyle=\footnotesize,
  numbers=left,
  numberstyle=\tiny,
  stepnumber=1,
  numbersep=5pt,
  showspaces=false,
  showstringspaces=false,
  showtabs=false,
  tabsize=4,
  captionpos=b,
  breaklines=true,
  breakatwhitespace=false,
  frame=single,
  frameround=tttt
}

\hypersetup{
    colorlinks,
    citecolor=black,
    filecolor=black,
    linkcolor=black,
    urlcolor=black
}

\newenvironment{packed_enum}{
\begin{enumerate}
  \setlength{\itemsep}{1pt}
  \setlength{\parskip}{0pt}
  \setlength{\parsep}{0pt}
}{\end{enumerate}}

\newenvironment{packed_item}{
\begin{itemize}
  \setlength{\itemsep}{1pt}
  \setlength{\parskip}{0pt}
  \setlength{\parsep}{0pt}
}{\end{itemize}}

\newenvironment{packed_desc}{
\begin{description}
  \setlength{\itemsep}{1pt}
  \setlength{\parskip}{0pt}
  \setlength{\parsep}{0pt}
}{\end{description}}

\title{Programming Assignment 4:\\An Encrypted Filesystem}
\author{
  CSCI 3753 - Operating Systems\\
  University of Colorado at Boulder\\
  Spring 2013
}
\date{\emph{Due Date: Friday, April 26th, 2013 11:55pm}}

\begin{document}

\maketitle

\section{Assignment Introduction}
In this assignment, we take a closer look at filesystems.
You will be writing a mirroring filesystem that provides a
transparent encryption wrapper on top of an exiting file system.
Specifically, you will be using FUSE\cite{fuse-website}
(Filesystems in USErspace) to implement your filesystem,
extended attributes (xattr) to differentiate between encrypted and
unencrypted files, and the OpenSSL\cite{openssl-website}
crypto\cite{openssl-evp} library to provide secure encryption.

The filesystem itself is a simple mirrored pass-through filesystem. For
example, if our pa4-encfs is mounted on the directory \textit{/tmp/pa4fs}
and is set to mirror the directory \textit{/home/user}, then any
action performed on the file \textit{/tmp/pa4fs/foo.txt}
will be translated into an action on the file
\textit{/home/user/foo.txt}.

Instead of just directly passing actions to the
underlying mirrored file, however, your pa4-encfs system will perform encryption
and decryption as necessary. Thus, if you create a new file
\textit{/tmp/pa4fs/bar.txt} in the aforementioned setup
and write some data to it, the data will
be encrypted before being written to the backing
\textit{/home/user/bar.txt} file. Likewise, if you were to read the
\textit{/tmp/pa4fs/bar.txt} file, your filesystem would read the backing encrypted
\textit{/home/user/bar.txt} file and decrypt the contents before
passing it to you. If you were to try to read
the backing \textit{/home/user/bar.txt} file directly, you would just
get encrypted binary gibberish. When your filesystem is unmounted,
your data will be securely stored in the mirrored directory
and indecipherable to anyone who may encounter it.

\section{Your Task}

The assignment is primarily a systems project. Thus, you will likely spend
more time pulling together and learning to assemble a diverse set of
existing APIs then you will spend writing actual code.

This work is best accomplished in a disciplined, iterative manner. Take
it one step at a time, and make sure each step is functioning before
moving on, and you will do well. Note that a working partial solution
is worth more credit then a broken ``full'' solution.

\subsection{Dependencies and Setup}
This assignment has several dependencies that must be installed in order
for the provided code to build correctly. On Ubuntu, start
by running \texttt{sudo apt-get update} to update your package
list. Then run \texttt{sudo apt-get install <package(s)>} to install the
following packages:

\begin{packed_item}
\item \texttt{fuse-utils}
\item \texttt{libfuse-dev}
\item \texttt{openssl}
\item \texttt{libssl-dev}
\item \texttt{libssl-doc} (optional)
\item \texttt{libssl1.0.0} or \texttt{libssl0.9.8}
\item \texttt{attr}
\item \texttt{attr-dev}
\end{packed_item}

You will need a working Internet connection in order to insure these
packages install correctly. Note that you can also specify multiple
packages in a single \texttt{sudo apt-get install <package(s)>}
call. Some packages may have their own dependencies, but
\texttt{apt-get} will automatically take care of installing these for you.

In addition to the dependencies discussed above, you may need to
enable extended attribute support on your system. This is done by
adding the \texttt{user\_xattr} mount option to any mount point on
which you wish to use extended attributes in your \texttt{/etc/fstab}
auto-mount configuration file. For example, on the Ubuntu 10.04 CS3753
VM, you must modify the following \texttt{/etc/fstab} line:

\begin{verbatim}
<file system>   <mount>  <type>  <options>          <dump>  <pass> 
...
UID=<Root UID>  /        ext4    errors=remount-ro  0       1
...
\end{verbatim}

so that it reads:

\begin{verbatim}
<file system>   <mount>  <type>  <options>                     <dump>  <pass> 
...
UID=<Root UID>  /        ext4    errors=remount-ro,user_xattr  0       1
...
\end{verbatim}

Note that Root UID will be unique to your system and that your
system may contain additional \texttt{/etc/fstab} lines that need
edited. Only data filesystem mounts need to be considered; swap, proc, and
other system mounts have no concept of extended attributes and should
be ignored. You must have root privileges to edit the
\texttt{/etc/fstab} file, so you will probably need to use
\texttt{sudo} in conjunction with the CLI editor of your choice.

\subsection{FUSE}

The FUSE API\cite{fuse-refs} is the core API used in this
assignment. It provides a means for implementing filesystem in
userspace. Normally filesystem are implemented as kernel modules, but
this reduces your ability to utilize userspace libraries. FUSE
provides a userspace interface for communicating with the FUSE kernel
module that performs the necessary in-kernel work on behalf of your
filesystem. Consult the Resources and References sections for additional
details.

All filesystem in Unix-like operating systems provide a common
interface to the user. We refer to this interface as the Virtual File
System (VFS)\cite{linux-vfs}. The VFS is what allows us to write
programs that can be completely agnostic to the underlying filesystems
on a given system. Like all Unix filesystem, a FUSE filesystem also
implements the VFS interface. The primary task involved in writing a
FUSE filesystem is to provide implementations for all of the functions
specified by the VFS.

To accomplish this task, your must implement a number of VFS functions,
and then pass function pointers to your implementations to the FUSE
system via a special struct. See the FUSE documentation and the included
FUSE examples for more information
\cite{fuse-website,fuse-refs,fuse-wiki,pfeiffer-fuse}.

\subsection{Encryption}

Encryption provides a means for us to mathematically protect data,
rendering it unreadable to everyone except those in possession of the
necessary encryption key. There is one primary rule to using
encryption in your own programs: \emph{Never code your own encryption
 algorithms}. Writing good encryption libraries is a very difficult
task. And only publicly reviewed, widely deployed, and heavily tested
libraries written by encryption experts should ever be trusted. No
encryption in preferable to bad encryption, as bad encryption just
encourages a false since of security.

Thus, we will be using the OpenSSL\cite{openssl-website} crypto
library\cite{openssl-evp} for our
assignment. The OpenSSL libraries are the de-facto standard for open
source encryption. In particular, we will be using the AES symmetric
encryption algorithm with a 256-bit key in CBC mode.

The \texttt{aes-crypt-util} demo program provides an example of using
AES encryption with a key derived from a user-provided pass phrase. The
encryption methods used in this example are also suitable for your
assignment.

The most common tripping point when using AES CBC
encryption in that encryption and decryption is an all-or-nothing
game. You can't encrypt or decrypt part of a file. You must encrypt or
decrypt the entire file or none of it. This is due to the fact that CBC
encryption schemes are state-full: the transform performed on each
encrypted block is a function of the transformed output of the previous
block. Thus, if we attempt to start encrypting or decrypting in
the middle of a file, we will wind up with data that can not be
used. You will need to consider this ``entire file or none of it''
limitation when designing your system.

Consult the Resources and References sections for additional
details. See the OpenSSL documentation and the included
\texttt{aes-crypt-util} example for more information
\cite{openssl-website,openssl-docs,openssl-evp,pillai-aes}.

\subsection{Extended Attributes}

The POSIX extended attribute systems (xattr) provides a means to
store additional file meta data beyond the standard time-stamp,
ownership, and permission data. It provides an interface for
associating arbitrary name-value pairs with specific files.

There are four operations (each with a corresponding system call) that
allow you to manipulate extend attributes: list, get, set, and
remove. See the \texttt{xattr-util} example file and the appropriate man
pages for details on the use of each function.

Extended attributes must be supported by the underlying filesystem to
function correctly. EXT2, EXT3, EXT4, XFS, ResierFS, and a number of
additional filesystem provide xattr support. Often, however, this
support is disabled by default and must be enabled via the
\texttt{user\_xattr} mount option in the \texttt{/etc/fstab} file. See
the previous Dependencies and Setup section for additional details on
configuring your system.

In addition, the Linux xattr implementation limits userspace use of
extended attributes to a specific ``user'' namespace. The namespace of an
extended attribute is defined by the first dot-separated field in its
name. Thus, all extended attributes modified from user space must begin
with a \texttt{user.} prefix. Failure to use this prefix will result in
an error when attempting to add, modify, or remove an extended
attribute. The included \texttt{xattr-util} program provides an
example of the standard means for transparently affixing this prefix
to all xattr actions. See \cite{freedesktop-xattr} for additional
xattr namespace information.

Consult the Resources and References sections for additional
details. See the xattr documentation, \texttt{attr} man page, and the included
texttt{xattr-util} example for more information.

\subsection{Development Tips}

As previously mentioned, you will be best served by taking an iterative
approach to this assignment. Start by familiarizing yourself with the
examples and available documentation. Then take your implementation one
step at a time, insure a single feature is working before moving on to
additional features. The following represents one possible feature development
order for this assignment:

\begin{packed_enum}
\item {\bf Start With a Copy of \texttt{fusexmp.c}}: This will provide
  you with the basics of a mirror file system on which to build.
\item {\bf Add Support for Specific Mirror Directory}:
  \texttt{fusexmp.c} only supports mirroring the root directory. Add an
  additional argument to \texttt{main} that allows the user to specify a specific
  directory to mirror (hint: \cite{pfeiffer-fuse} contains information
  of passing FUSE private data via \texttt{fuse\_main} function 
  and accessing it within your VFS implementations).
  Then modify each VFS function to redirect
  actions to the corresponding file in the specified mirror directory
  (hint: this involves a
  rather simple transformation of the \texttt{path} variable that many
  VFS functions are passed).
\item {\bf Add Support for Encryption}: Add encryption and decryption
  support to your system. Remember that files must be encrypted and
  decrypted in a single aes-crypt pass. Also, note that you can
  use the \texttt{aes-crypt-util} function to manually encrypt or decrypt files
  to aid you in testing your system. Just make sure to use the same
  key pass phrase when encrypting or decrypting manually as you pass to
  your filesystem.
\item {\bf Add Support for an XATTR Encrypted Flag} Update the flag
  when necessary and only perform crypto operations when it indicates
  that a file in encrypted. Again, you may find the provided
  \texttt{xattr-util} function useful when testing or debugging your
  system.
\end{packed_enum}

If you encounter errors, you may find FUSE's debugging capabilities
handy. If you run a FUSE mount executable with the \texttt{-d} flag, it
will mount the filesystem in debug mode. In this mode, all FUSE status
is printed to the terminal. Use one terminal instance to monitor FUSE
and another to interact with your filesystem and force specific
behaviors and actions. Print output sent to \texttt{stderr} from
within your implementation is also available via this debug mode.

\section{Requirements and Specifications}

In addition to the requirements and specification elicited elsewhere in
this document, your file system must satisfy the following in order to
receive full credit.

\begin{packed_item}
  \item {\bf Executable} Your filesystem should build a standard FUSE
    executable file called \texttt{pa4-encfs} that mounts your file
    system. Your executable should accept the following call format:
    \texttt{./pa4-encfs <Key Phrase> <Mirror Directory> <Mount Point>}.
  \item {\bf Read Behavior} When an encrypted file is read through your
    filesystem, it should be transparently decrypted and the plaintext
    data passed to the reading application. When an unencrypted file
    is read through your filesystem, the data should be passed directly
    to the reading application.
\item {\bf Write Behavior} When an existing encrypted file is written through your
    filesystem, the plaintext data passed from the writing application
    should be transparently encrypted before being written to the
    final destination in the mirror directory.
    When an existing unencrypted file
    is written through your filesystem, the plaintext data passed
    from the writing application should be written directly to the
    final destination in the mirror directory.
\item {\bf Create Behavior} When a new file is created through your
  filesystem, it should be encrypted (even empty files) and flagged as
  such.
\item {\bf Encryption Strength} Encryption should, at a minimum, meet
  AES 256-bit CBC security levels. The provided \texttt{aes-crypt.h}
  functions meet this requirement.
\item {\bf Extended Attributes} Your filesystem should store a flag
  indicating whether or not a file is encrypted using the the extended
  attributes for the corresponding mirror file. The flag should reside
  in the \texttt{user} namespace\cite{freedesktop-xattr} and be named
  \texttt{pa4-encfs.encrypted}. It should have ascii text values of either
  ``true'' or ``false''. If no flag is detected on a specific file,
  then it should be assumed that the file is not encrypted (same
  behavior as detecting a flag with a value equal to ``false'').
\item {\bf Supported Functions} At a minimum your filesystem must
  support the functions included in the \texttt{fusexmp}
  example. Additional functions may be implemented if you deem them
  necessary for your purposes.
\end{packed_item}

\section{What's Included}

We provide some code and examples to help get you started.
Feel free to use it as a jumping off point (appropriately cited).

\begin{packed_item}
\item {\bf Makefile} A GNU Make makefile to build all the code listed
  here.
\item {\bf README} As the title so eloquently instructs: read
  it. Provides usage instructions and examples for files listed here.
\item {\bf fusehello.c} A basic "Hello World" FUSE example. See
  \texttt{README} for usage instructions.
\item {\bf fusexmp.c} A basic FUSE mirrored filesystem example that
  mirrors the root directory (/) and supports most standard
  operations. See \texttt{README} for usage instructions.
\item {\bf xattr-util.c} A basic extended attribute manipulation
  program. Provides an example of proper Linux xattr use.
  See \texttt{README} for usage instructions.
\item {\bf aes-crypt-util.c} A basic AES encryption program using the
  local aes-crypt library (see \texttt{aes-crypt.h}) and the OpenSSL EVP
  API\cite{openssl-evp}. See \texttt{README} for usage instructions.
\item {\bf aes-crypt.h} A basic AES file-pointer centric encryption
  library interface. Implemented in \texttt{aes-crypt.c}.
\item {\bf aes-crypt.c} A basic AES file-pointer centric encryption
  library implementation. Uses the OpenSSL EVP API\cite{openssl-evp}.
\end{packed_item}

\section{What You Must Provide}

When you submit your assignment, you must provide the following as a
single archive file:
\begin{packed_enum}
\item A copy of your pa4-endfs FUSE code
\item A copy of any supporting code used by your filesystem
\item A makefile that builds any necessary code
\item A README explaining how to build and run your code
\end{packed_enum}

\section{Grading}

40\% of you grade will be based on implementing a filesystem that
meets the following criteria. You will be expected to provide
functional proof of the following criteria during your grading
session.

\begin{packed_item}
\item {\bf +10 points:} Filesystem properly mirrors target directory
  specified at mount time.
\item {\bf +10 points:} Filesystem uses extended attributes to differentiate
  between encrypted and unencrypted files.
\item {\bf +10 points:} Filesystem can transparently read and write
  securely encrypted files using a pass phrase specified at mount time.
\item {\bf +10 points:} Filesystem can transparently read and update unencrypted
  files
\end{packed_item}

In addition, the following items are worth extra credit. In no case
will the maximum score on the assignment exceed 110/100.
\begin{packed_item}
\item {\bf +10 extra points:} Filesystem encrypts, hides, or
  otherwise obfuscates the directory structure to make it
  cryptographically difficult to determine the names or locations of
  encrypted files.
\item {\bf +10 extra points:} Filesystem encrypts, hides, or
  otherwise obfuscates file attributes to make it
  cryptographically difficult to determine the time-stamps, ownership,
  and permissions of encrypted files.
\item {\bf +5 extra points:} Filesystem supports multiple encryption
  keys for different files,
  using an additional extended attributes to mark each file with
  a unique identifier indicating the key used to encrypt it. Filesystem only
  attempts to decrypt files associated with the currently loaded
  key. You can think of this as a basic form of multi-user support.
\item {\bf +5 extra points:} Filesystem supports multiple encryption
  keys for a single file. You will probably need to use a layered
  encryption method, where a master key is used to encrypt each file,
  and then multiple copies of the master key are themselves encrypted
  with the necessary additional keys and stored for future
  retrieval. Such a system would be useful if you wished to provide
  multiple users with access to an encrypted file with forfeiting your
  master key. When combined with the previous item, this will implement
  fairly versatile multi-user support.
\end{packed_item}

If your code does not build or run without errors, you will not receive
any credit on the objective portion (40\%) of your assignment.

If your code generates warnings when building under gcc on the VM
using \texttt{-Wall} and \texttt{-Wextra} you will be penalized 1
point per warning. In addition, to receive full credit your submission must:
\begin{packed_item}
\item Meet all requirements elicited in this document
\item Code must adhere to good coding practices.
\item Code must be submitted to Moodle prior to due date.
\end{packed_item}

The other 60\% of your grade will be determined via your grading
interview where you will be expected to explain your work and answer
questions regarding it and any concepts related to this assignment.

\section{Obtaining Code}
The starting code for this assignment is available on the Moodle and
on github. If you would like practice using a version control system,
consider forking the code from github. Using the github code is not
a requirement, but it will help to insure that you stay up to date
with any updates or changes to the supplied codebase. It is also
good practice for the kind of development one might expect to do in
a professional environment. And since your github code can be easily
shared, it can be a good way to show off your coding skills to
potential employers and other programmers.

Github code may be forked from the project page here:\\
\url{https://github.com/cgartrel/CU-CS3753-PA4}.

\section{Resources}
Refer to your textbook and class notes on the Moodle for an overview
of filesystems.

If you require a good C language reference, consult K\&R\cite{K+R}. If
you need an updated C99 reference see Harbison \& Steele\cite{H+S}.

The Internet\cite{tubes} is also a good resource for finding
information related to solving this assignment.

You may wish to consult the man pages for the following items, as they
will be useful and/or required to complete this assignment. Note that
the first argument to the ``man'' command is the chapter, insuring
that you access the appropriate version of each man page. See
\texttt{man 1 man} for more information. Not all of these man pages are
installed be default. Install the previously discussed dependencies or
consult an online man page repository if you can not locate a specific
man page on your system.

\begin{packed_item}
\item \texttt{man 1 make}
\item \texttt{man 1 fusermount}
\item \texttt{man 5 attr}
\item \texttt{man 2 setxattr}
\item \texttt{man 2 getxattr}
\item \texttt{man 2 listxattr}
\item \texttt{man 2 removexattr}
\item \texttt{man 3 EVP}
\item \texttt{man 3 EVP\_CipherUpdate}
\item \texttt{man 3 crypto}
\item Many of the system calls used in the FUSE examples also have man pages
\end{packed_item}

In addition, you may find a number of the references in the bibliography helpful.

\begin{thebibliography}{13}

\bibitem{freedesktop-xattr} freedesktop.org.
  \newblock \emph{Guidelines for extended attributes}.
  \newblock \url{http://www.freedesktop.org/wiki/CommonExtendedAttributes}.

\bibitem{fuse-website} FUSE.
  \newblock \emph{Filesystems in Userspace}.
  \newblock \url{http://fuse.sourceforge.net/}.

\bibitem{fuse-refs} FUSE.
  \newblock \emph{Fuse Doxygen API Reference}.
  \newblock \url{http://fuse.sourceforge.net/doxygen/index.html}.

\bibitem{fuse-wiki} FUSE.
  \newblock \emph{Fuse Wiki}.
  \newblock \url{http://sourceforge.net/apps/mediawiki/fuse/index.php?title=Main_Page}.

\bibitem{H+S} Harbison, Samuel and Steele, Guy.
  \newblock \emph{C: A Reference Manual}.
  \newblock Fifth Edition: 2002.
  \newblock Prentice Hall: New Jersey.

\bibitem{linux-vfs} Johnson, Michael.
  \newblock \emph{A tour of the Linux VFS}.
  \newblock 1996.
  \newblock \url{http://tldp.org/LDP/khg/HyperNews/get/fs/vfstour.html}.

\bibitem{K+R} Kernighan, Brian and Dennis, Ritchie.
  \newblock \emph{The C Programming Language}.
  \newblock Second Edition: 1988.
  \newblock Prentice Hall: New Jersey.

\bibitem{openssl-website} OpenSSL.
  \newblock \emph{Cryptography and SSL/TLS Toolkit}.
  \newblock \url{http://www.openssl.org/}.

\bibitem{openssl-docs} OpenSSL.
  \newblock \emph{OpenSSL Documents}.
  \newblock \url{http://www.openssl.org/docs/}.

\bibitem{openssl-evp} OpenSSL.
  \newblock \emph{OpenSSL EVP Documentation}.
  \newblock \url{http://www.openssl.org/docs/crypto/EVP_EncryptInit.html}.

\bibitem{pillai-aes} Pillai, Saju.
  \newblock \emph{Openssl AES encryption example}.
  \newblock Decemper 9th, 2008.
  \newblock \url{http://saju.net.in/blog/?p=36}.

\bibitem{pfeiffer-fuse} Pfeiffer, Joseph.
  \newblock \emph{Writing a FUSE Filesystem: a Tutorial}.
  \newblock January 10th, 2011.
  \newblock \url{http://www.cs.nmsu.edu/~pfeiffer/fuse-tutorial/}.

\bibitem{tubes} Stevens, Ted.
  \newblock \emph{Speech on Net Neutrality Bill}.
  \newblock 2006.
  \newblock \url{http://youtu.be/f99PcP0aFNE}.

\end{thebibliography}

\end{document}  
